\documentclass{article}
\usepackage[spanish]{babel}
\author{Carlos}
\title{Resumen Heroes of Math Island version {\LaTex} } 
\date{16 de agosto 2022}
\begin{document}

\maketitle

\section{Introducción y contexto}
    Para el paper los juegos de video son potencial tecnologias de aprendizaje avazando (ALT en ingles). Se cuestiona por qué hay tan bajo uso de los video juegos como plataforma de aprendizaje. Se llega a concluir que una de las razones es que la mayoría de juegos de aprendizaje, no llegan a ser igual de divertidos que los juegos de ocio. Las preguntas que se hacen el paper son las siguentes:
\newline 
\newline 


1)¿ Cuales son las reacciones subjetivas de los estudiantes jugando Heros of Math Island. ?
\newline 
\newline 
2)¿ Cuales son las respuestas emocionales y el agente emocional ?
\newline 
\newline 
3)¿ Cuales son los intereses y logros en matemáticas en el grupo test despues de haber jugado el juego ?
\newline 
\newline 


    Cognición, emociones,motivación, estética, comunicación, interaciones sociales, sociología y tecnología son todas las variables que se identifican  a tener en cuenta en el desdarrolo del juego, para que exista un impacto en el aprendizaje.
\newline 
\newline 


    Hereos of Math Island, es un juego pensado para estudiantes de quinto y septimo. Hay una isla con 5 busquedas (Quest), y es necesario ir pasando por una en una (Bosque,mina, monataña, etc.). El juego tiene un sistema de refozamiento inteligente, o sea que va detectando los errorres de las personas y reforzando. Tambien el juego se pone más dificil si se aciertan varias respuestas consecutivamente. El agente emocional es un monito que sale al costado de la pantalla.
\newline 
\newline 


    Las emociones fueron sacadas de literiatura de affecting computing, al final se dejaron 12 más las emonción neutral: aburrimiento, confianza, confusión/nerviosismo, curiosidad, placer, decepción, alta concentración, exitación, frustación, orgullo, culpa y sorpresa. El monito tiene 6 emociones, la neutra, felicidad, confianza, triste y frustado.La intención es que el monito sea el agente emocional, y ver que tanto impacta que el monito esté en pantalla en el aprendizaje.  
\newline 
\newline 


\section{métodos y técnicas}
    Comentario: Dejé esto acá por si falta tiempo, dudo que se pueda poner, lo considere bastante sencillo y explicable en otras partes de este documento.
\newline 
\newline 


\section{Resultados de la investigación}
    A las pequeñas criatures les gustó mucho el juego, de las mayorías de las preguntas generadas para el experimento, muy pocas tuvierón respuestas negativas. La mayoría quería que le agregaran más busquedas (Quest), y que le agregaran ladder, recompenzas, items coleccionables y cambiar el diseño de avatar. Caracteristicas propias de los juegos de ocio. 
\newline 
\newline 


    La gran mayoría de las pequeñas criaturas les gustó el juego, al punto que lo preferían más que los libros de texto algunos. En el fondo, todas las persona salvo una, vieron el juego como una aumentación de su aprendizaje, la cual en muchos casos el juego fue preferido más que la clase tradicional. Existió una diferencia significativa en cuanto al pre test y post test de lo aprendido.
\newline 
\newline 
   
    Comentario: Mañana pongo los datos del pre test y post test aquí :) 
\newline 
\newline 


    De las emociones, la que más resalto fue la felicidad, una persona dijo que se sintió realizada en la vida a la hora de haber jugado el juego.  Cabe señalar que una persona con dificultad de atención para las matemáticas, mencionó que durante el juego sintió que su concentración estaba al máximo, cuando por lo general no era así.
\newline 
\newline 


    La segunda emoción resaltada fue la de autoconfianza, la tercera confución. Se señala que la confución es sinergica, ya que es tomada positivamente. La cuarta emoción fue frustación. Se menciona que la frustación es reducida por el factor del juego, o sea que las personas aprenden con una mayor resiliencia por estar en algo lúdico. En el juego si te equivocas mucho te mandan a un mapa distinto donde tienes que redimirte, el paper señala que este reforzamiento negativo tuvo efectos negetavios en algunas personas, por lo que se dice que no es un buen método de enseñanza.
\newline 
\newline 


    El 20  porciento de las personas le desagradó el monito. Las razones eran porque desconcentraba de la realización de los ejercicios. A algunas personas les desagradó la interacción con el monito, por lo general pensaba que estaba demas o que no era de mucha utilidad. A un 26.6 porciento no le gustó el monito, y a un 53.3 porciento no lo concideró molesto.
\newline 
\newline 

\section{conclusión}
    Se observa que el agente emocional (el monito) tuvo un rechazo, y las personas entregaron ideas de como mejorar a este. El muestreo de 15 personas es muy poco para ser concluyente a gran estacala (Meta analisis). Pero Lo que si queda en claro es que el agente emocional puede ser de gran impacto, pero debe ser diseñado minuciosamente. Esto es complicado por la heterogeneidad de las personas, donde el agente emocional puede tener distintas respuesta, dependiendo del background cultural y con las cosas que se identifican. Lo cual lo hace un problema sociológico a la hora de su diseño.  

\end{document}

